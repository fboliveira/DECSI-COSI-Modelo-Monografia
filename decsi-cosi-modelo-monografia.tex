%% -----------------------------------------------------------------------
%%
%% MODELO DE MONOGRAFIA DO DECSI
%%
%% -----------------------------------------------------------------------
%% Versions and updates:
%% = Glauber M. Cabral, glauber@decsi.ufop.br (v1.0)
%% = Oto Braz Assunção, oto.braz@outlook.com (v1.1)
%% 		- Adequações à nova resolução do COSI.
%% = Fernando B, Oliveira, fernando@decea.ufop.br (v1.2)
%% 		- Correções na Folha de Aprovação.
%%		- Mudanças na capa
%% 		- Inclusão do Termo de Responsabilidade e da Ata de Defesa atualizada.
%% 		- Estrutura do documento LaTeX
%% = Contribuições: COSI e Biblioteca JM
%% -----------------------------------------------------------------------
%% ESTE MODELO FOI BASEADO EM:
%% -----------------------------------------------------------------------
%% abtex2-modelo-trabalho-academico.tex, v-1.9.6 laurocesar
%% Copyright 2012-2016 by abnTeX2 group at http://www.abntex.net.br/
%%
%% This work may be distributed and/or modified under the
%% conditions of the LaTeX Project Public License, either version 1.3
%% of this license or (at your option) any later version.
%% The latest version of this license is in
%%   http://www.latex-project.org/lppl.txt
%% and version 1.3 or later is part of all distributions of LaTeX
%% version 2005/12/01 or later.
%%
%% This work has the LPPL maintenance status `maintained'.
%%
%% The Current Maintainer of this work is the abnTeX2 team, led
%% by Lauro César Araujo. Further information are available on
%% http://www.abntex.net.br/
%%
%% This work consists of the files abntex2-modelo-trabalho-academico.tex,
%% abntex2-modelo-include-comandos and abntex2-modelo-references.bib
%%
% ------------------------------------------------------------------------
% ------------------------------------------------------------------------
% abnTeX2: Modelo de Trabalho Academico (tese de doutorado, dissertacao de
% mestrado e trabalhos monograficos em geral) em conformidade com
% ABNT NBR 14724:2011: Informacao e documentacao - Trabalhos academicos -
% Apresentacao
% ------------------------------------------------------------------------
% ------------------------------------------------------------------------

\documentclass[
  % -- opções da classe memoir --
  12pt,       % tamanho da fonte
  openright,      % capítulos começam em pág ímpar (insere página vazia caso preciso)
  oneside,      % para impressão em verso e anverso. Oposto a oneside
  a4paper,      % tamanho do papel.
  % -- opções da classe abntex2 --
  %chapter=TITLE,   % títulos de capítulos convertidos em letras maiúsculas
  %section=TITLE,   % títulos de seções convertidos em letras maiúsculas
  %subsection=TITLE,  % títulos de subseções convertidos em letras maiúsculas
  %subsubsection=TITLE,% títulos de subsubseções convertidos em letras maiúsculas
  % -- opções do pacote babel --
  english,      % idioma adicional para hifenização
  french,        % idioma adicional para hifenização
  spanish,     % idioma adicional para hifenização
  brazil        % o último idioma é o principal do documento
  ]{abntex2-decsi}

% ----------------------------------------------------------
% PACOTES E CONFIGURAÇÕES GERAIS
%
% Insira no arquivos os pacotes utilizados e as respectivas configurações.
%
% ----------------------------------------------------------
% Pacotes e Configurações
% ----------------------------------------------------------

% ---
% Pacotes básicos
% ---
\usepackage{enumitem}
\usepackage{lmodern}      % Usa a fonte Latin Modern
\usepackage[T1]{fontenc}    % Selecao de codigos de fonte.
\usepackage[utf8]{inputenc}   % Codificacao do documento (conversão automática dos acentos)
\usepackage{lastpage}     % Usado pela Ficha catalográfica
\usepackage{indentfirst}    % Indenta o primeiro parágrafo de cada seção.
\usepackage{color}        % Controle das cores
\usepackage{graphicx}     % Inclusão de gráficos
\usepackage{microtype}    % para melhorias de justificação
\usepackage{float}
\usepackage{caption}
\usepackage{subcaption}
% ---

% ---
% Pacotes adicionais, usados apenas no âmbito do Modelo Canônico do abnteX2
% ---
\usepackage[geometry]{ifsym}
\usepackage{multicol}
\usepackage{pdfpages}
\usepackage[printonlyused]{acronym} % pacote para produzir acrônimos - utilize [printonlyused] para gerar a lista somente com os que foram utilizados

% ---

% ---
% Pacotes de citações
% ---
\usepackage[brazilian,hyperpageref]{backref}   % Paginas com as citações na bibl
\usepackage[alf]{abntex2cite} % Citações padrão ABNT

% FBO: Review
\usepackage{color}
\newcommand{\review}[1]{{\textbf{\color{red}{#1}}}}

\newenvironment{reviewblock}
{\bfseries\color{red}}
{\normalfont\color{black}}

% ---
% CONFIGURAÇÕES DE PACOTES
% ---

% ---
% Configurações do pacote backref
% Usado sem a opção hyperpageref de backref
\renewcommand{\backrefpagesname}{Citado na(s) página(s):~}
% Texto padrão antes do número das páginas
\renewcommand{\backref}{}
% Define os textos da citação
\renewcommand*{\backrefalt}[4]{
  \ifcase #1 %
    Nenhuma citação no texto.%
  \or
    Citado na página #2.%
  \else
    Citado #1 vezes nas páginas #2.%
  \fi}%
% ---

% ---

% Tamanho da linha de assinatura segundo o tamanho do maior nome
% Sugestão: 8cm
\setlength{\ABNTEXsignwidth}{8cm}

% ---
% Configurações de aparência do PDF final

% alterando o aspecto da cor azul
\definecolor{blue}{RGB}{41,5,195}

% informações do PDF
\makeatletter
\hypersetup{
      %pagebackref=true,
    pdftitle={\@title},
    pdfauthor={\@author},
      pdfsubject={\imprimirpreambulo},
      pdfcreator={LaTeX with abnTeX2},
    pdfkeywords={abnt}{latex}{abntex}{abntex2}{trabalho acadêmico},
    colorlinks=true,          % false: boxed links; true: colored links
      linkcolor=blue,           % color of internal links
      citecolor=blue,           % color of links to bibliography
      filecolor=magenta,          % color of file links
    urlcolor=blue,
    bookmarksdepth=4
}
\makeatother
% ---

% ---
% Espaçamentos entre linhas e parágrafos
% ---

% O tamanho do parágrafo é dado por:
\setlength{\parindent}{1.3cm}

% Controle do espaçamento entre um parágrafo e outro:
\setlength{\parskip}{0.2cm}  % tente também \onelineskip

% ---
% compila o indice
% ---
\makeindex
% ---

% ----------------------------------------------------------

% ----------------------------------------------------------
% DADOS PESSOAIS, COMPOSIÇÃO DA BANCA E DEFESA
%
% Insira no arquivos os dados acerca do trabalho,
% da composição da banca e da defesa.
%
% ---
% Informações de dados para CAPA e FOLHA DE ROSTO
% ---
\titulo{Modelo de Trabalho de Conclusão de Curso}
\autor{Nome Completo do Aluno}
\local{João Monlevade}
\dia{11}
\mes{3}
\ano{2019} % Indicar apenas o ano da monografia

% ---
% Dados da Universidade e Curso
% ---
\instituicao{Universidade Federal de Ouro Preto}
\instituto{Instituto de Ciências Exatas e Aplicadas}
\departamento{Departamento de Computação e Sistemas}
\colegiado{Colegiado de Sistemas de Informação}
\curso{Sistemas de Informação}
\grau{Bacharel em Sistemas de Informação}
\disciplina{CSI499 -- Trabalho de Conclusão de Curso II}

% ---
% Dados de Orientação e Banca de Defesa
% ---
\orientador{Nome Completo do Orientador}
\orientadorTitulacao{Titulação}
\orientadorDepartamento{DECSI -- UFOP}

\coorientador{Nome Completo do Coorientador}
\coorientadorTitulacao{Titulação}
\coorientadorDepartamento{DECSI -- UFOP}

\convidadoUm{Nome Completo do Convidado1}
\convidadoUmTitulacao{Titulação}
\convidadoUmDepartamento{DECSI -- UFOP}

\convidadoDois{Nome Completo do Convidado2}
\convidadoDoisTitulacao{Titulação}
\convidadoDoisDepartamento{DECSI -- UFOP}

\convidadoTres{Nome Completo do Convidado3}
\convidadoTresTitulacao{Titulação}
\convidadoTresDepartamento{DECSI -- UFOP}

% ---
% Informações sobre a data de defesa - ata
% ---
\localdefesa{C304}
\horadefesa{17:00}

% ---
% Tipo de Trabalho
% ---
\tipotrabalho{Monografia (graduação)}

% ----------------------------------------------------------

% ----
% Início do documento
% ----
\begin{document}

% Seleciona o idioma do documento (conforme pacotes do babel)
%\selectlanguage{english}
\selectlanguage{brazil}

% Retira espaço extra obsoleto entre as frases.
\frenchspacing

% ----------------------------------------------------------
% ELEMENTOS PRÉ-TEXTUAIS
% ----------------------------------------------------------
% \pretextual

% ---
% Capa
% ---
\imprimircapa%
% ---

% ---
% Folha de rosto
% (o * indica que haverá a ficha bibliográfica)
% ---
\imprimirfolhaderosto*
% ---

% ---
% Inserir a ficha bibliografica
% ---
% Ficha catalográfica: elaborada pela biblioteca (opcional para TCC)
% Será impressa no verso da folha de rosto e não deverá ser contada.

% Isto é um exemplo de Ficha Catalográfica, ou ``Dados internacionais de
% catalogação-na-publicação''. Você pode utilizar este modelo como referência.
% Porém, provavelmente a biblioteca da sua universidade lhe fornecerá um PDF
% com a ficha catalográfica definitiva após a defesa do trabalho.

% Quando estiver com o documento, salve-o como PDF no diretório do seu projeto e substitua todo o conteúdo de implementação deste arquivo pelo comando abaixo:
%
% \begin{fichacatalografica}
%     \includepdf{fig_ficha_catalografica.pdf}
% \end{fichacatalografica}

 \begin{fichacatalografica}
   %\vspace*{\fill}         % Posição vertical
   \vspace*{10cm}
   \hrule              % Linha horizontal
   \begin{center}          % Minipage Centralizado
   \begin{minipage}[c]{12.5cm}   % Largura
     A Ficha Catalográfica é elaborada exclusivamente pela Biblioteca. Substitua esta página pelo documento gerado na versão final da sua monografia.
   \end{minipage}
   \end{center}
   \hrule
 \end{fichacatalografica}

 \pagebreak

% ---

% ---
% Inserir errata, caso necessário
% ---
\include{./pre-textuais/errata}
% ---

% ---
% Inserir folha de aprovação
% ---

% Isto é um exemplo de Folha de aprovação, elemento obrigatório da NBR
% 14724/2011 (seção 4.2.1.3). Você pode utilizar este modelo até a aprovação
% do trabalho. Após isso, substitua todo o conteúdo deste arquivo por uma
% imagem da página assinada pela banca com o comando abaixo:
%
% \includepdf{folhadeaprovacao_final.pdf}
%

\imprimirfolhadeaprovacao%
% ---

% ---
% Ata de defesa
% ---

% Isto é um exemplo de ata de defesa, um documento obrigatório conforme a Resolução do COSI.
% Você pode utilizar este modelo até a aprovação do trabalho.
% Após isso, substitua todo o conteúdo deste arquivo por uma
% imagem da página assinada pela banca com o comando abaixo:
%
% \includepdf{ata-de-defesa.pdf}
%

\imprimiratadedefesa%
% ---

% ---
% Termo de Responsabilidade
% ---

% Isto é um exemplo de termo de responsabilidade, um documento obrigatório conforme a Resolução do COSI.
% Você pode utilizar este modelo até a aprovação do trabalho.
% Após isso, substitua todo o conteúdo deste arquivo por uma
% imagem da página assinada pela banca com o comando abaixo:
%
% \includepdf{termo-de-responsabilidade.pdf}
%

\imprimirtermoderesponsabilidade%
% ---

%\hrule
\pagebreak

% ---
% Dedicatória
% ---
\begin{dedicatoria}
   \vspace*{\fill}
   \centering
   \noindent
   \textit{Este trabalho é dedicado ...} \vspace*{\fill}
\end{dedicatoria}
% ---

% ---

% ---
% Agradecimentos
% ---
\begin{agradecimentos}

Agradeço ...

\end{agradecimentos}
% ---

% ---
% Epígrafe
% ---
\begin{epigrafe}
    \vspace*{\fill}
  \begin{flushright}

	\textit{``Science is more than a body of knowledge; it is a way of thinking.''} \\ ~ \\
	--- Carl Sagan (1934 -- 1996), \\ \textit{in: The Demon-Haunted World: Science as a Candle in the Dark.}

  \end{flushright}
\end{epigrafe}
% ---

% ---
% RESUMOS
% ---
% resumo em português
\include{./pre-textuais/resumo}

% resumo em inglês
\include{./pre-textuais/abstract}
% ---

% ---
% inserir lista de ilustrações
% ---
\pdfbookmark[0]{\listfigurename}{lof}
\listoffigures*
\cleardoublepage%
% ---

% ---
% inserir lista de tabelas
% ---
\pdfbookmark[0]{\listtablename}{lot}
\listoftables*
\cleardoublepage%
% ---

% ---
% inserir lista de abreviaturas e siglas
% ---
% Este é o modelo do abnTeX2
% ---
% inserir lista de abreviaturas e siglas
% ---
%\begin{siglas}
%  \item[ABNT] Associação Brasileira de Normas Técnicas
%  \item[abnTeX] ABsurdas Normas para TeX
%\end{siglas}
% ---

%
% Outra opção é utilizar o pacote Acronym (mais funcional).
%

%\ac{SW}  	SW documents are beautifully typeset.
%\acf{SW} 	Scientific Word (SW) documents are beautifully typeset.
%\acs{SW} 	SW documents are beautifully typeset.
%\acl{SW} 	Scientific Word documents are beautifully typeset.

\pdfbookmark[0]{\listadesiglasname}{los}
\chapter*{\listadesiglasname}%

% Incluir no sumário
% \addcontentsline{toc}{chapter}{\listadesiglasname}%

\begin{acronym}
  \acro{COSI}{Colegiado do Curso de Sistemas de Informação}
  \acro{DECSI}{Departamento de Computação e Sistemas}
  \acro{ICEA}{Instituto de Ciências Exatas e Aplicadas}
  \acro{JM}{João Monlevade}
  \acro{SI}{Sistemas de Informação}
  \acro{UFOP}{Universidade Federal de Ouro Preto}
	\acro{VRP}{\textit{Vehicle Routing Problem}}
\end{acronym}

% ---

% ---
% inserir lista de símbolos, caso se aplique.
% ---
\include{./pre-textuais/listas/simbolos}
% ---

% ---
% inserir o sumario
% ---
\pdfbookmark[0]{\contentsname}{toc}
\tableofcontents*
\cleardoublepage{}
% ---

% ----------------------------------------------------------
% ELEMENTOS TEXTUAIS
% ----------------------------------------------------------
\textual{}


% ----------------------------------------------------------
% Capítulos
% ----------------------------------------------------------

% ---
% Introdução
% ---
% ----------------------------------------------------------
% Introdução
% ----------------------------------------------------------
\chapter{Introdução}
\label{cap:introducao}
% ----------------------------------------------------------

Este documento é um modelo de monografia para o Curso de \acf{SI}. Este curso é vinculado ao \acf{DECSI} do \acf{ICEA} da \acf{UFOP}. O modelo foi elaborado de acordo com a Resolução n$^\circ$ 012 do \acf{COSI} de 7 de março de 2016 (com atualização em 18 de outubro de 2016). Para a utilização deste documento em outros cursos recomenda-se a verificação junto ao colegiado em questão acerca do modelo apropriado. Contribuições importantes ao modelo foram feitas pelo COSI e pela equipe da Biblioteca de \acf{JM}.

O modelo é uma versão personalizada da classe \abnTeX\ e utilizada de acordo como a licença associada (\textit{LaTeX Project Public License v1.3c}\footnote{\url{https://www.latex-project.org/lppl/}}). Informações sobre a classe \abnTeX\ podem ser obtidas em \url{http://www.abntex.net.br/} e \url{https://github.com/abntex/abntex2}.

Para facilitar a organização, os itens foram separados de acordo com a estrutura definida pela norma \citeonline{NBR14724:2011}. Os grupos principais são \textit{pré-textuais}, \textit{textuais} e \textit{pós-textuais}, como apresentado a seguir:

\begin{verbatim}
		+ decsi-cosi-modelo-monografia.tex
		| + pre-textuais -> dedicatoria, agradecimentos, epígrafe, resumos,
		|										dentre outros.
		| + textuais -> capítulos da monografia.
		| + pos-textuais -> apêndices e anexos.
\end{verbatim}

As demais pastas foram incluídas como apoio aos itens, além de conter arquivos complementares.

\begin{verbatim}
		| + bib -> arquivo de referência bibliográfica - bibtex.
		| + documentos -> resoluções COSI e normas.
		| + config -> dados e pacotes.
		| + img -> imagens e afins.
\end{verbatim}

Recomenda-se a leitura de referências sobre metodologia científica para a definição correta da estrutura dos capítulos da monografia. Dentre outras, sugere-se a leitura de \citeonline{wazlawick:2009} e \citeonline{SilvaMenezes:2005}.

Seguem algumas sugestões de estrutura para os capítulos e recomendações sobre a escrita. \textbf{É importante observar que entre as seções deve ter um texto introdutório. Um tópico nunca deverá ficar sem um texto relacionado a ele} (Biblioteca JM). Termos e expressões em outras línguas devem estar em itálico: \textit{Model, View, Controller, Database}, dentre outros. Acerca das \textbf{fontes} em imagens, tabelas, gráficos e demais itens, mesmo que esses itens sejam gerados pelo(a) aluno(a), \textbf{é necessário incluir}: ``Fonte: elaborado pelo autor'' ou ``Fonte: dados da pesquisa'' (Biblioteca JM). Você pode encontrar também diversos exemplos de utilização do modelo que foram elaborados pela Equipe do \abnTeX\ no Anexo \ref{cap_exemplos}. As instruções para compilação do documento são apresentadas na Seção \ref{abntex:compila}.

% ----------------------------

\section{Elaboração do capítulo}

Este capítulo apresenta o seu trabalho. Você deve contextualizar o problema abordado, descrever os objetivos gerais e específicos, apresentar a metodologia e como o trabalho está estruturado.

\section{O problema de pesquisa}
\label{sec:problema}

O problema de pesquisa

\section{Objetivos}
\label{sec:objetivos}

O presente trabalho consiste ...

Este trabalho possui aos seguintes objetivos específicos: (\textbf{utilizar os verbos no infinitivo}).

\begin{itemize}
	\item Desenvolver ...
	\item Incorporar ...
	\item Validar
\end{itemize}

\section{Metodologia}
\label{sec:metodologia}

O objeto de pesquisa deste trabalho ...

Os passos para execução deste trabalho são assim definidos:

\begin{itemize}
	\item Revisão da literatura ..
	\item Desenvolvimento ...
	\item Validação ...
	\item Análise e discussão ...
\end{itemize}

\section{Organização do trabalho}

\textbf{É importante observar que a estrutura é apresentada a partir do próximo capítulo. O capítulo de Introdução não deve compor esta descrição. Além disso, sempre que você fizer referência à algum item específico, a inicial deve ser maiúscula. Por exemplo, Capítulo 2, Tabela 5, Figura 1, dentre outros.}

O restante deste trabalho é organizado como se segue. O Capítulo~\ref{cap:revisao} apresenta...


% ----------------------------------------------------------
% PARTE
% ----------------------------------------------------------
%\part{Referenciais teóricos}
% ----------------------------------------------------------
% ----------------------------------------------------------
% Capitulo de revisão de literatura
% ----------------------------------------------------------
\chapter{Revisão bibliográfica}
\label{cap:revisao}
% ----------------------------------------------------------

Este capítulo apresenta uma revisão da literatura, bem como trabalhos correlatos.


% ----------------------------------------------------------
% PARTE
% ----------------------------------------------------------
%\part{Desenvolvimento}
% ----------------------------------------------------------
% ----------------------------------------------------------
% Desenvolvimento
% ----------------------------------------------------------
\chapter{Desenvolvimento}
\label{cap:desenvolvimento}
% ----------------------------------------------------------

Este capítulo descreve o desenvolvimento do trabalho...


% ----------------------------------------------------------
% PARTE
% ----------------------------------------------------------
%\part{Resultados}
% ----------------------------------------------------------
% ----------------------------------------------------------
% Capítulo de Resultados
% ----------------------------------------------------------
\chapter{Resultados}
\label{cap:resultados}
% ---

Primeiro capítulo de resultados do trabalho.


% ----------------------------------------------------------
% Finaliza a parte no bookmark do PDF
% para que se inicie o bookmark na raiz
% e adiciona espaço de parte no Sumário
% ----------------------------------------------------------
\phantompart{}

% ----------------------------------------------------------
% Conclusão
% ----------------------------------------------------------
% ----------------------------------------------------------
% Conclusão
% ----------------------------------------------------------
\chapter[Conclusão]{Conclusão}
%\addcontentsline{toc}{chapter}{Conclusão}
% ---

Este trabalho apresentou...

% ---

% ----------------------------------------------------------
% ELEMENTOS PÓS-TEXTUAIS
% ----------------------------------------------------------
\postextual%
% ----------------------------------------------------------

% ----------------------------------------------------------
% Referências bibliográficas
% ----------------------------------------------------------
\bibliography{./bib/decsi-cosi-modelo-monografia}

% ----------------------------------------------------------
% Glossário
% ----------------------------------------------------------
%
% Consulte o manual da classe abntex2 para orientações sobre o glossário.
%
%\glossary

% ----------------------------------------------------------
% Apêndices
% ----------------------------------------------------------

% ---
% Inicia os apêndices
% ---
\begin{apendicesenv}

% Imprime uma página indicando o início dos apêndices
\partapendices%

% ---
% Primeiro apendice
% ----------------------------------------------------------
\chapter{Materiais elaborados pelo autor}
\label{cap:apendice}
% ----------------------------------------------------------

Apêndices são os materiais elaborados pelo autor, ou seja, com objetivo de completar uma argumentação (Biblioteca JM).

% ---

% ---
% Demais apendices
% % ----------------------------------------------------------
\chapter{Materiais elaborados pelo autor}
\label{cap:apendice}
% ----------------------------------------------------------

Apêndices são os materiais elaborados pelo autor, ou seja, com objetivo de completar uma argumentação (Biblioteca JM).

% ---


\end{apendicesenv}
% ---


% ----------------------------------------------------------
% Anexos
% ----------------------------------------------------------

% ---
% Inicia os anexos
% ---
\begin{anexosenv}

% Imprime uma página indicando o início dos anexos
\partanexos%

% ---
% Primeiro anexo
% ---
\chapter{Outros materiais}
\label{cap:anexo}
% ---

Anexos são materiais não elaborados pelo autor, que servem de fundamentação, comprovação e ilustração (Biblioteca JM).

% ---

% ---
% Capitulo com exemplos de comandos inseridos de arquivo externo
% ---
\include{./pos-textuais/anexos/abntex2-modelo-include-comandos}
% ---

% ---
% Demais anexos
% % ---
\chapter{Outros materiais}
\label{cap:anexo}
% ---

Anexos são materiais não elaborados pelo autor, que servem de fundamentação, comprovação e ilustração (Biblioteca JM).

% ---

\end{anexosenv}

%---------------------------------------------------------------------
% INDICE REMISSIVO
%---------------------------------------------------------------------
\phantompart%
\printindex
%---------------------------------------------------------------------

\end{document}
